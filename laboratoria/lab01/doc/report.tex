\documentclass{article}

\usepackage[OT4,plmath]{polski}
\usepackage[T1]{fontenc}
\usepackage[utf8]{inputenc}
\usepackage{hyperref}
\usepackage{indentfirst}
\usepackage{graphicx}
\usepackage{textcomp}
\usepackage{parskip}
\usepackage{url}
\usepackage{fancyvrb}
%\usepackage[fixlanguage]{babelbib}
%\selectbiblanguage{polish}
%\setlength{\parindent}{24pt}
%\linespread{1.1}

\frenchspacing

\newcommand{\labnumber}{1}
\newenvironment{ttblock}{\ttfamily}{\par}

%------------------------------------------------------------------------------

\title{Linux w systemach wbudowanych\\Laboratorium \#\labnumber}

\author{\href{mailto:bezap@student.mini.pw.edu.pl}{Patryk \textsc{Bęza}}, \href{mailto:spychalam@student.mini.pw.edu.pl}{Maciej \textsc{Spychała}}}

\date{\today}

\begin{document}

\maketitle

\begin{center}
\begin{tabular}{ll}
Grupa laboratoryjna: & A\\
Data laboratorium \labnumber A: & 17 III 2016\\
Data laboratorium \labnumber B: & 24 III 2016\\
Wykładowca: & \href{mailto:wzab@ise.pw.edu.pl}{dr inż. Wojciech Zabołotny}
\end{tabular}
\end{center}

%\begin{abstract}
%Abstract text
%\end{abstract}

%------------------------------------------------------------------------------

\section{Zadanie}
\label{task}

Zadaniem laboratorium było skonfigurowanie \emph{\href{https://buildroot.org/}{Buildroot}'a} i \emph{\href{https://www.raspberrypi.org/}{Raspberry~Pi}} tak, aby \emph{Raspberry~Pi}:

\begin{enumerate}
\item Automatycznie łączyło się z siecią, używając protokołu DHCP do otrzymania parametrów sieci. Jeśli kabel sieciowy nie jest podłączony, połączenie powinno zostać zestawione automatycznie po podłączeniu kabla. Odłączenie kabla powinno powodować wyłączenie połączenia.
\item Miało przydzieloną nazwę systemu w postaci \texttt{nazwisko1\_nazwisko2}, czyli w tym przypadku: \texttt{Beza\_Spychala}.
\item Automatycznie aktualizowało czas systemowy, wykorzystując do tego klienta NTP (najlepiej z wykorzystaniem \emph{NTP server pool}).
\item Miało zainstalowany dowolny serwer HTTP udostępniający statyczne strony WWW, np.~serwer~\emph{Lighthttpd} (alternatywnie: \emph{nginx} lub \emph{Apache}).
\item Miało poza użytkownikiem \texttt{root}, użytkownika domyślnego, którego hasło jest ustawiane przy starcie systemu.
\end{enumerate}

%------------------------------------------------------------------------------

\section{Konfiguracja}

Poniżej opisano kroki jakie trzeba podjąć, aby skonfigurować \emph{Buildroot'a} i \emph{Raspberry~Pi} tak, aby spełnić wymagania zadania opisane w rozdziale~\ref{task}.

%------------------------------------------------------------------------------

\subsection{Konfiguracja wstępna}

Przed rozpoczęciem konfigurowania \emph{Buildroot'a}, w taki sposób, aby spełniał warunki zadania, należy skonfigurować podstawowe parametry \emph{Buildroot'a} zgodnie z \href{http://wzab.cba.pl/LINSW/poradnik\_laboratorium.pdf}{poradnikiem konfiguracji}, przygotowanym przez Wykładowcę~\cite{www:wzab}, tzn. należy:

\begin{enumerate}
\item Pobrać środowisko \emph{Buildroot} w najnowszej dostępnej wersji, tj.~\texttt{2016.02}:

\begin{ttblock}
\$ wget https://buildroot.org/downloads/buildroot-2016.02.tar.bz2
\end{ttblock}

\item Rozpakować pobrane środowisko:

\begin{ttblock}
\$ tar -xjf buildroot-2016.02.tar.bz2
\end{ttblock}

\item Wstępnie skonfigurować do działania z płytką \emph{Raspberry~Pi}:

\begin{ttblock}
\$ cd buildroot-2016.2\\
\$ make raspberrypi\_defconfig
\end{ttblock}

\item Uruchomić \texttt{make menuconfig}.

\item Ustawić TTY opcją \texttt{BR2\_TARGET\_GENERIC\_GETTY\_PORT}:

\begin{ttblock}
System configuration \textrightarrow\ Run a getty (login prompt) after boot \textrightarrow\ TTY port \textrightarrow\ ttyAMA0
\end{ttblock}

\item Dla poprawy szybkości pobierania zależności oraz dla uniknięcia zablokowania IP laboratoryjnego \emph{gateway'a} przez serwery, z których grupowo pobieramy pakiety, należy ustawić IP lokalnego serwera serwującego część potrzebnego oprogramowania ustawiając opcję \texttt{BR2\_PRIMARY\_SITE}:

\begin{ttblock}
Build options \textrightarrow\ Mirrors and Download locations \textrightarrow\ Primary download site \textrightarrow\ http://192.168.137.24/dl
\end{ttblock}

\item Wyłączyć obsługę modelu \emph{'hard' floating point} opcją \texttt{BR2\_ARM\_EABI}:

\begin{ttblock}
Target options \textrightarrow\ Target ABI \textrightarrow\ EABI
\end{ttblock}

\item Zmienić \emph{Toolchain} opcjami \texttt{BR2\_TOOLCHAIN\_EXTERNAL} i\\ \texttt{BR2\_TOOLCHAIN\_EXTERNAL\_CODESOURCERY\_ARM}:

\begin{ttblock}
Toolchain \textrightarrow\ Toolchain type \textrightarrow\ External toolchain\\
Toolchain \textrightarrow\ Toolchain \textrightarrow\ Sourcery CodeBench ARM 2014.05
\end{ttblock}

\item Dodać \texttt{initramfs} jako rodzaj generowanego obrazu \emph{filesystem'u}, pozostawiając zaznaczony również \texttt{ext2/3/4 variant (ext4)} -- opcja \\\texttt{BR2\_TARGET\_ROOTFS\_INITRAMFS}:

\begin{ttblock}
Filesystem images \textrightarrow\ initial RAM filesystem linked into linux kernel
\end{ttblock}
\end{enumerate}

Po wykonaniu powyższych kroków, możemy przejść do realizacji właściwego zadania laboratorium, co opisano w kolejnych podrozdziałach.

%------------------------------------------------------------------------------

\subsection{Konfiguracja sieci}

Aby włączyć obsługę \href{https://www.ietf.org/rfc/rfc2131.txt}{DHCP}, należy:
\begin{enumerate}
\item Wybrać opcję \texttt{BR2\_PACKAGE\_DHCPCD}:

\begin{ttblock}
Target packages \textrightarrow\ Networking applications \textrightarrow\ dhcpcd
\end{ttblock}

\item Włączyć opcję \texttt{BR2\_SYSTEM\_DHCP}, aby podczas \emph{bootowania}, system używał DHCP do pobrania parametrów sieci:

\begin{ttblock}
System configuration \textrightarrow\ Network interface to configure through DHCP
\end{ttblock}

\item Wybrać pakiet \emph{\href{http://linux.die.net/man/8/ifplugd}{ifplugd}}, aby system automatycznie łączył się z siecią przy wkładaniu i wyjmowaniu kabla sieciowego do portu RJ45 -- opcja \texttt{BR2\_PACKAGE\_IFPLUGD}:

\begin{ttblock}
Target packages \textrightarrow\ Networking applications \textrightarrow\ ifplugd
\end{ttblock}

lub alternatywnie wybrać \emph{\href{http://linux.die.net/man/8/netplugd}{netplugd}} wybierając opcję \texttt{BR2\_PACKAGE\_NETPLUG}:

\begin{ttblock}
Target packages \textrightarrow\ Networking applications \textrightarrow\ netplug
\end{ttblock}

\item Stworzyć \emph{filesystem overlay} (opcja \texttt{BR2\_ROOTFS\_OVERLAY}) w postaci 3 plików tekstowych:
\begin{enumerate}
	\item \texttt{/etc/ifplugd/ifplugd.action}
	\item \texttt{/etc/ifplugd/ifplugd.conf}
	\item \texttt{/etc/init.d/ifplugd}
\end{enumerate}
Powyższe pliki zostały dołączone do niniejszego raportu jako załącznik.
\end{enumerate}

\noindent W przypadku gdyby \emph{Raspberry~Pi} korzystało do połączenia z internetem z USB, należałoby skorzystać np. z pakietu \emph{\href{https://wiki.archlinux.org/index.php/Eudev}{eudev}}:
\begin{enumerate}
\item Włączyć pakiet~\texttt{eudev} wybierając opcję \texttt{BR2\_PACKAGE\_EUDEV}:

\begin{ttblock}
System configuration \textrightarrow\ /dev management\ \textrightarrow\ Dynamic using devtmpfs + eudev
\end{ttblock}

\item Zaznaczyć opcję \texttt{BR2\_PACKAGE\_EUDEV\_RULES\_GEN}:

\begin{ttblock}
Target packages \textrightarrow\ Hardware handling \textrightarrow\ eudev \textrightarrow\\Enable rules generator
\end{ttblock}

\item Zaznaczyć opcję \texttt{BR2\_PACKAGE\_EUDEV\_ENABLE\_HWDB}:

\begin{ttblock}
Target packages \textrightarrow\ Hardware handling \textrightarrow\ eudev \textrightarrow\ Enable hwdb installation
\end{ttblock}
\end{enumerate}

\noindent Dzięki włączeniu \texttt{eudev}, moglibyśmy skonfigurować automatyczne łączenie z siecią dodając regułę do \href{http://www.reactivated.net/writing_udev_rules.html}{\texttt{/etc/udev/rules.d}}~\cite{www:arch-udev}. Po dodaniu takich reguł należałoby uruchomić \texttt{\href{http://linux.die.net/man/8/udevadm}{udevadm} control -{}-reload-rules}, aby system załadował dodane reguły.

%------------------------------------------------------------------------------

\subsection{Nazwa systemu}

Aby zmienić nazwę systemu należy wypełnić opcję \texttt{BR2\_TARGET\_GENERIC\_HOSTNAME}:

\begin{center}
\texttt{System configuration \textrightarrow\ System hostname}
\end{center}

\noindent i ustawić własną nazwę systemu -- w naszym przypadku: \texttt{Beza\_Spychala}.

%------------------------------------------------------------------------------

\subsection{NTP}

Aby włączyć automatyczną aktualizację czasu systemu przy pomocy protokołu \href{http://www.ntp.org/rfc.html}{NTP}, należy włączyć \emph{daemona} NTP wybierając opcję \texttt{BR2\_PACKAGE\_NTP}:

\begin{center}
\texttt{Target packages \textrightarrow\ Networking applications \textrightarrow\ ntp \textrightarrow\ ntpd}
\end{center}

%------------------------------------------------------------------------------

\subsection{Serwer HTTP}

Aby zainstalować serwer HTTP, serwujący proste, statyczne strony WWW, włączamy opcję aktywującą~\emph{\href{https://www.lighttpd.net/}{lighthttpd}} \texttt{BR2\_PACKAGE\_LIGHTTPD}:

\begin{center}
\texttt{Target packages \textrightarrow\ Networking applications \textrightarrow\ lighthttpd}
\end{center}

oraz dodajemy następujące pliki jako \emph{filesystem overlay}:
\begin{itemize}
\item plik konfiguracyjny \texttt{/etc/lighttpd/lighttpd.conf} z \href{https://redmine.lighttpd.net/projects/1/wiki/TutorialConfiguration}{podstawową konfiguracją} serwera HTTP:
\begin{Verbatim}[frame=single]
server.document-root = "/var/www/public/"

server.port = 80

mimetype.assign = (
  ".html" => "text/html",
  ".txt" => "text/plain",
  ".jpg" => "image/jpeg",
  ".png" => "image/png"
\end{Verbatim}
\item plik \texttt{/var/www/public/index.html}, serwowany przez serwer HTTP.
\end{itemize}

%------------------------------------------------------------------------------

\subsection{Użytkownicy systemu}
\label{uzyszkodnik}

Aby dodać użytkownika do systemu, wypełniamy opcję \texttt{BR2\_ROOTFS\_USERS\_TABLES} wchodząc do:

\begin{center}
\texttt{System configuration \textrightarrow\ Path to the users tables}
\end{center}

\noindent i podajemy ścieżkę do pliku ze zdefiniowaną tablica użytkowników systemowych w formacie \href{https://buildroot.org/downloads/manual/manual.html#makeuser-syntax}{opisanym} w dokumentacji \emph{Buildroot'a}~\cite{www:makeuser-syntax}. Ścieżka ta zostanie przekazana jako parametr do~\href{https://github.com/maximeh/buildroot/blob/master/support/scripts/mkusers}{\texttt{mkusers}}.

\noindent Przykładowa treść pliku \texttt{users.txt}, definiującego dodanego użytkownika:

\begin{Verbatim}[frame=single]
uzyszkodnik -1 users -1 =password /home/uzyszkodnik /bin/sh -
\end{Verbatim}

%------------------------------------------------------------------------------

\section{Uruchamianie}

Aby uruchomić \emph{Raspberry~Pi} z wyżej opianą konfiguracją, należy uruchomić kompilację \emph{Buildroot'a}, a następnie załadować tak skompilowany system na kartę pamięci SD, włożoną do slotu \emph{Raspberry~Pi}, podłączonego do sieci kablem \emph{ethernetowym}.

\subsection{Kompilacja}

W celu rozpoczęcia budowania systemu przeznaczonego dla \emph{Raspberry~Pi}, należy zachować zmiany poczynione w trakcie konfiguracji \emph{Buildroot'a} i wydać komendę~\texttt{\href{https://github.com/buildroot/buildroot/blob/master/Makefile}{make}}.

%------------------------------------------------------------------------------

\subsection{Ładowanie do pamięci \emph{Raspberry~Pi}}

W celu załadowania nowego obrazu systemu do pamięci \emph{Raspberry~Pi}, należy:
\begin{enumerate}
\item Zamontować pierwszą partycję karty pamięci SD:
\begin{itemize}
\item \texttt{mkdir /sd}
\item \texttt{mount /dev/mmcblk0p1 /sd}
\end{itemize}
\item Usunąć stare jądro: \texttt{rm /sd/zImage}.
\item Włączyć interfejs sieciowy \texttt{eth0} \emph{Raspberry~Pi}: \texttt{\href{http://linux.die.net/man/8/ifconfig}{ifconfig} eth0 up}.
\item Uruchomić \emph{daemona} DHCP komendą \texttt{\href{http://dev.man-online.org/man8/udhcpc/}{udhcpc}}, aby ustalił parametry połączenia sieciowego.
\item Uruchomić serwer SSH \texttt{\href{https://matt.ucc.asn.au/dropbear/dropbear.html}{dropbear}}, zainstalowany na karcie SD, włożonej do slotu płytki \emph{Raspberry~Pi}.
\item Przekopiować programem \texttt{\href{http://linux.die.net/man/1/scp}{scp}} nowy obraz jądra \texttt{zImage} na kartę SD.
\item Zrestartować \emph{Raspberry~Pi} np. komendą \texttt{\href{http://linux.die.net/man/8/reboot}{reboot}}.
\end{enumerate}

%------------------------------------------------------------------------------

\section{Załącznik}

Do niniejszego raportu załączono archiwum:
\begin{center}
\texttt{lab01\_BezaSpychala.tar.gz},
\end{center}
o sumie kontrolnej MD5:
\begin{center}
\texttt{5ebbbfa1b66cfc10fde75f1dca87e1de}
\end{center}
które jest minimalnym zestawem plików, pozwalającym odtworzyć i zweryfikować projekt zgodnie z niżej opisaną procedurą:
\begin{enumerate}
\item W dowolnym katalogu, np. w \texttt{/tmp/test}, rozpakować \emph{Buildroot'a}:
\begin{center}
\texttt{tar -xjvf buildroot-2016.02.tar.bz2}
\end{center}

Dla ustalenia uwagi, niech rozpakowany \emph{Buildroot} będzie w~katalogu:
\begin{center}
\texttt{/tmp/test/buildroot\_2016.2}
\end{center}

\item W katalogu \texttt{/tmp/test} należy rozpakować archiwum \texttt{lab01\_BezaSpychala.tar.gz}:
\begin{center}
\texttt{tar -xzvf lab01\_BezaSpychala.tar.gz}
\end{center}

Po rozpakowaniu powstanie katalog plik z konfiguracją \emph{Buildroot'a}:
\begin{center}
\texttt{/tmp/test/.config}\\
\end{center}
oraz katalog:
\begin{center}
\texttt{/tmp/test/config},
\end{center}
w którym znajdują się:
	\begin{enumerate}
		\item plik \texttt{users.txt} ze zdefiniowanym użytkownikiem \texttt{uzyszkodnik}, którego treść przytoczono w rozdziale~\ref{uzyszkodnik},
		\item katalog \texttt{overlay}, który służy jako \emph{\href{https://buildroot.org/downloads/manual/manual.html\#customize}{filesystem overlay}} -- znajdują się w nim konfiguracja \texttt{ifplugd}, \texttt{lighthttpd} oraz przykładowa, statyczna strona WWW.
	\end{enumerate}
\item Przenieść plik \texttt{/tmp/test/.config} do katalogu \texttt{/tmp/test/buildroot\_2016.2}
\item Będąc w katalogu \texttt{/tmp/test/buildroot\_2016.2} uruchomić \texttt{make}
\end{enumerate}

Katalog \texttt{/tmp/test/config} celowo pozostał poza katalogiem \emph{Buildroot'a}, tzn. poza katalogiem \texttt{/tmp/test/buildroot\_2016.2}, ponieważ w pliku:
\begin{center}
\texttt{/tmp/test/buildroot\_2016.2/.config}
\end{center}
została ustawiona opcja:
\begin{center}
\texttt{BR2\_ROOTFS\_OVERLAY="../config/overlay"}
\end{center}

%------------------------------------------------------------------------------

\bibliographystyle{plain}
\bibliography{bibliography}

\end{document}
