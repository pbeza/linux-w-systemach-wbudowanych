\documentclass{article}

\usepackage[OT4,plmath]{polski}
\usepackage[T1]{fontenc}
\usepackage[utf8]{inputenc}
\usepackage{hyperref}
\usepackage{indentfirst}
\usepackage{graphicx}
\usepackage{textcomp}
\usepackage{parskip}
\usepackage{url}
\usepackage{fancyvrb}
\usepackage{dirtree}
\usepackage{listings}
%\usepackage[fixlanguage]{babelbib}
%\selectbiblanguage{polish}
%\setlength{\parindent}{24pt}
%\linespread{1.1}

\frenchspacing

\newcommand{\labnumber}{3}
\newenvironment{ttblock}{\ttfamily}{\par}

%------------------------------------------------------------------------------

\title{Linux w systemach wbudowanych\\Laboratorium \#\labnumber}

\author{\href{mailto:bezap@student.mini.pw.edu.pl}{Patryk \textsc{Bęza}}, \href{mailto:spychalam@student.mini.pw.edu.pl}{Maciej \textsc{Spychała}}}

\date{\today}

\begin{document}

\maketitle

\begin{center}
\begin{tabular}{ll}
Grupa laboratoryjna: & A\\
Data laboratorium \labnumber A: & 28 IV 2016\\
Data laboratorium \labnumber B: & 5 V 2016\\
Wykładowca: & \href{mailto:wzab@ise.pw.edu.pl}{dr inż. Wojciech Zabołotny}
\end{tabular}
\end{center}

%\begin{abstract}
%Abstract text
%\end{abstract}

%------------------------------------------------------------------------------

\section{Zadanie}
\label{task}

Zadaniem laboratorium było:
\begin{enumerate}
\item Przygotować ,,administracyjny'' system Linux pracujący w \texttt{initramfs}, umożliwiający przygotowanie karty pamięci SD (partycjonowanie, formatowanie, przesłanie przez sieć nowej wersji systemu) do instalacji systemu Linux pracującego z systemem plików \texttt{ext2}~(3 lub~4) montowanym z partycji 2 na karcie SD. Spartycjonować i sformatować kartę tak, aby partycje 2 i 3 zawierały system plików \texttt{ext2}, \texttt{ext3} lub \texttt{ext4} (partycję 1 zostawić).
\item Przygotować ,,użytkowy'' system Linux pracujący z systemem plików \texttt{e2fs}, zawierający serwer WWW, udostępniający pliki z partycji 3 na karcie SD (powinna być ładnie prezentowana lista dostępnych plików) i umożliwiający wgrywanie nowych plików po podaniu nazwy użytkownika i hasła (nie musi to być użytkownik i hasło systemowe).
\item Należy też przygotować \emph{bootloader}, umożliwiający określenie (przy pomocy przycisku), który system ma zostać załadowany. Użytkownik powinien być powiadomiony diodą LED, kiedy jest badany stan przycisków (np. z sekundowym wyprzedzeniem). Jeśli żaden przycisk nie zostanie wciśnięty, powinien zostać załadowany system ,,użytkowy''. Diody LED powinny informować o tym, jaki system został wybrany.
\end{enumerate}

Plan pracy:
\begin{enumerate}
	\item \textbf{(3pkt.)} Przygotować początkową wersję systemu administracyjnego. Upewnić się, że zawiera on narzędzia niezbędne do:
	\begin{enumerate}
		\item przesłania plików przez sieć,
		\item zarządzania kartą SD (formatowanie, tworzenie systemów plików -- \texttt{vfat} i \texttt{e4fs}, naprawa systemu plików, rozpakowanie archiwum z systemem pliku itp.),
		\item narzędzia niezbędne do stworzenia skryptu bootloadera (\texttt{mkimage}),
		\item wygodny edytor.
	\end{enumerate}
	\item \textbf{(1pkt.)} Odpowiednio spertycjonować kartę SD. Jeśli partycja 1 zostanie zmieniona, należy zachować oryginalną zawartość partycji \texttt{VFAT} w ramdysku, tak aby można było ją przywrócić po utworzeniu i sformatowaniu nowych partycji.
	\item \textbf{(3pkt.)} Utworzyć skrypt \emph{bootloadera}. Sprawdzić, że pozwala on wybrać ładowany system (np. ładując system ,,administracyjny'' z różnymi parametrami).
	\item \textbf{(7pkt.)} Przygotować system ,,użytkowy''. Pamiętać, że ma on używać systemu plików \texttt{e4fs}. Dlatego, aby go zainstalować należy:
		\begin{enumerate}
			\item skopiować plik \texttt{zImage} z jądrem,
			\item skopiować skompresowane archiwum \texttt{tar} z obrazem systemu plików i rozpakować je na drugiej partycji karty SD (oczywiście po jej zamontowaniu!),
			\item sugeruje się zachowanie dwóch niezależnych środowisk \emph{Buildroot} -- jednego dla systemu ,,administracyjnego'' i drugiego dla systemu ,,użytkowego'', tak, żeby można było szybko dokonywać korekt w obu systemach.
		\end{enumerate}
		Ponadto należy spełnić następujące wymagania:
		\begin{enumerate}
			\item System ,,użytkowy'' jest poprawnie ładowany. Interfejs \texttt{eth0} jest poprawnie skonfigurowany. Partycje systemowa i danych są poprawnie zamontowane po sprawdzeniu i ewentualnej naprawie systemu plików.
			\item Serwer WWW poprawnie udostępnia pliki z trzeciej partycji.
			\item Serwer WWW pozwala wgrać nowy plik po uwierzytelnieniu użytkownika.
		\end{enumerate}
		Należy pamiętać, że serwer WWW zrealizowany na bazie \emph{Tornado} można łatwo uruchamiać i testować na maszynie wirtualnej. \texttt{QEMU} pozwala emulować system z emulowanym dyskiem twardym lub kartą SD.
\end{enumerate}

%------------------------------------------------------------------------------

\section{Konfiguracja}

Poniżej opisano kroki jakie trzeba podjąć, aby skonfigurować \emph{Buildroot'a} i \emph{Raspberry~Pi} tak, aby spełnić wymagania zadania opisane w rozdziale~\ref{task}.

%------------------------------------------------------------------------------

\subsection{Konfiguracja wstępna}

Przed rozpoczęciem konfigurowania \emph{Buildroot'a}, w taki sposób, aby spełniał warunki zadania, należy skonfigurować podstawowe parametry \emph{Buildroot'a} zgodnie z \href{http://wzab.cba.pl/LINSW/poradnik\_laboratorium.pdf}{poradnikiem konfiguracji}, przygotowanym przez Wykładowcę~\cite{www:wzab}, tzn. należy:

\begin{enumerate}
\item Pobrać środowisko \emph{Buildroot} w najnowszej dostępnej wersji, tj.~\texttt{2016.02}:

\begin{ttblock}
\$ wget https://buildroot.org/downloads/buildroot-2016.02.tar.bz2
\end{ttblock}

\item Rozpakować pobrane środowisko:

\begin{ttblock}
\$ tar -xjf buildroot-2016.02.tar.bz2
\end{ttblock}

\item Wstępnie skonfigurować do działania z płytką \emph{Raspberry~Pi}:

\begin{ttblock}
\$ cd buildroot-2016.2\\
\$ make raspberrypi\_defconfig
\end{ttblock}

\item Uruchomić \texttt{make menuconfig}.

\item Ustawić TTY opcją \texttt{BR2\_TARGET\_GENERIC\_GETTY\_PORT}:

\begin{ttblock}
System configuration \textrightarrow\ Run a getty (login prompt) after boot \textrightarrow\ TTY port \textrightarrow\ ttyAMA0
\end{ttblock}

\item Dla poprawy szybkości pobierania zależności oraz dla uniknięcia zablokowania IP laboratoryjnego \emph{gateway'a} przez serwery, z których grupowo pobieramy pakiety, należy ustawić IP lokalnego serwera serwującego część potrzebnego oprogramowania ustawiając opcję \texttt{BR2\_PRIMARY\_SITE}:

\begin{ttblock}
Build options \textrightarrow\ Mirrors and Download locations \textrightarrow\ Primary download site \textrightarrow\ http://192.168.137.24/dl
\end{ttblock}

\item Wyłączyć obsługę modelu \emph{'hard' floating point} opcją \texttt{BR2\_ARM\_EABI}:

\begin{ttblock}
Target options \textrightarrow\ Target ABI \textrightarrow\ EABI
\end{ttblock}

\item Zmienić \emph{Toolchain} opcjami \texttt{BR2\_TOOLCHAIN\_EXTERNAL} i\\ \texttt{BR2\_TOOLCHAIN\_EXTERNAL\_CODESOURCERY\_ARM}:

\begin{ttblock}
Toolchain \textrightarrow\ Toolchain type \textrightarrow\ External toolchain\\
Toolchain \textrightarrow\ Toolchain \textrightarrow\ Sourcery CodeBench ARM 2014.05
\end{ttblock}

%\item Dodać \texttt{initramfs} jako rodzaj generowanego obrazu \emph{filesystem'u}, pozostawiając zaznaczony również \texttt{ext2/3/4 variant (ext4)} -- opcja \\\texttt{BR2\_TARGET\_ROOTFS\_INITRAMFS}:
%
%\begin{ttblock}
%Filesystem images \textrightarrow\ initial RAM filesystem linked into linux kernel
%\end{ttblock}
\end{enumerate}

Powyższa konfiguracja wstępna dotyczy systemu zarówno dla systemu administracyjnego jak i użytkowego. Kolejne kroki konieczne do zrealizowania zadania niniejszego laboratorium opisano w kolejnych podrozdziałach.

%------------------------------------------------------------------------------

\subsection{Partycjonowanie karty SD}
\label{partitions}

Na początku należało sformatować kartę SD, tak, aby partycja:
\begin{itemize}
	\item \textbf{Pierwsza} -- która zawiera pliki potrzebne do ząładowania systemu, w tym m.in. plik obrazu \emph{U-boot'a} \texttt{boot.scr} oraz 2 obrazy jąder Linux'a: dla systemu administracyjnego i użytkowego. Partycja ta jest sformatowna z wykorzystaniem systemu plików \texttt{vfat}.
	\item \textbf{Druga} -- która zawiera system administracyjny. Sformatowana z wykorzystaniem systemu plików \texttt{ext4}.
	\item \textbf{Trzecia} -- która zawiera system użytkownika. Sformatowana z wykorzystaniem systemem plików \texttt{ext4}.
\end{itemize}

%------------------------------------------------------------------------------

\subsection{Konfiguracja bootloader'a U-boot}

Jako \emph{bootloader} został użyty \href{http://www.denx.de/wiki/u-boot/}{\emph{U-boot}}, którego konfiguracja sprowadza się do napisania skryptu \texttt{boot.txt}, z którego, za pomocą narzędzia \href{http://linux.die.net/man/1/mkimage}{\texttt{mkimage}}, można wygenerować plik \texttt{boot.scr}, który jest obrazem dla \emph{U-boot'a}, ładowanym w czasie startu \emph{RaspberryPi}. Komenda to wygenerowania obrazu \texttt{boot.scr}:
\begin{verbatim}
mkimage -T script -C none -n 'Start script' -d boot.txt boot.scr
\end{verbatim}

Dzięki \emph{bootloader'owi}, za pomocą przycisku na dołączonej płytce, mamy możliwość wyboru ładowania systemu administracyjnego lub użytkowego. Jeśli w czasie rozruchu systemu żaden przycisk nie zostanie wciśnięty, domyślnie zostanie załadowany system użytkowy, co jest sygnalizowane zapaleniem się zielonej lampki na dołączonej płytce.

W przeciwnym razie, jeśli w czasie nieprzekraczającym \emph{timeout'u}, równym 1~sekundę, przycisk zostanie wciśnięty, zostanie załadowany system administracyjny. Co ważne, dzięki \href{https://www.kernel.org/doc/Documentation/filesystems/ramfs-rootfs-initramfs.txt}{\texttt{initramfs}}, system administracyjny zostanie załadowany do RAMu, dzięki czemu, mamy możliwość dokonywania niemal dowolnych zmian na wszystkich partycjach w czasie pracy administratora.

Treść pliku \texttt{boot.txt}:
\begin{lstlisting}[frame=single,breaklines,basicstyle=\small]
gpio toggle 17;
sleep 1;
gpio toggle 17;
if gpio input 10; then
        gpio toggle 24;
        setenv bootargs "${bootargs_orig} console=ttyAMA0"
        load mmc 0:1 $fdt_addr_r bcm2708-rpi-b.dtb
        load mmc 0:1 $kernel_addr_r zImage
        bootz $kernel_addr_r - $fdt_addr_r
fi;

gpio toggle 23;

fatload mmc 0:1 ${kernel_addr_r} zImage.std
setenv bootargs "${bootargs_orig} console=ttyAMA0 root=/dev/mmcblk0p2 rootwait"
bootz ${kernel_addr_r}
\end{lstlisting}
Jądro \texttt{zImage.std} to jądro systemu zwykłego użytkownika, a \texttt{zImage} to jądro administratora. \texttt{sleep 1} odpowiada za uśpienie \emph{bootloader'a} na 1~sekundę, tak, aby użytkownik miał czas na ewentualne wciśnięcie przycisku odpowiadającego za uruchomienie systemu administracyjnego.

%------------------------------------------------------------------------------

\subsection{Konfiguracja systemu użytkowego}

Aby skonfigurować system użytkowy zgodnie z wymaganiami przedstawionymi w rozdziale~\ref{task}, należy w konfiguracji \emph{Buildroot'a} (tj. po wydaniu komendy np.~\texttt{make menuconfig}) wybrać:
\begin{itemize}
	\item \emph{Python'a} -- \texttt{BR2\_PACKAGE\_PYTHON},
	\item Serwer WWW \emph{Tornado} -- \texttt{BR2\_PACKAGE\_PYTHON\_TORNADO}.
\end{itemize}
Ponadto należy w \emph{Buildroot'cie} ustawić \emph{Filesystem overlay}, dzięki któremu system automatycznie nawiąże połączenie z siecią, uruchomi \texttt{fsck} dla partycji \texttt{/dev/mmcblk0p3}, którą następnie zamontuje, a na koniec uruchomi serwer \emph{Tornado}, listujący HTML'owo pliki dostępne na tymże serwerze.

\emph{Overlay} znajduje się w spakowanym załączniku do niniejszego raportu w katalogu \texttt{/overlay}. Należy zadbać o to, aby prawa wykonania do pliku \path{/overlay/etc/init.d/S50filesrv} pliku były dostatecznie wysokie (wywołując np. komendę \texttt{chmod +x S50filesrv}).

W katalogu \path{/config/overlay} znajduje się również kod odpowiedzialny za wyświetlanie listy plików na serwerze, pobieranie plików z serwera i ładowanie plików na serwer, tj. pliki: \path{/config/overlay/root/server.py} oraz \path{/config/overlay/root/static/index.html}. Pliki te zostały te dostosowane do własnych potrzeb na podstawie \emph{Tornado File Lister'a}, opublikowanego na wolnej licencji \emph{MIT} na \emph{GitHub'ie}~\cite{www:file-lister}.

Po skompilowaniu obrazu jądra systemu użytkowego i skopiowaniu go na pierwszą partycję karty SD pod nazwą \texttt{zImage.std}, należy jeszcze skopiować zawartość zbudowanego archiwum \texttt{rootfs.tar} na trzecią partycję karty SD (uprzednio ją formatując -- patrz \ref{partitions}). W archiwum tym znajduje się wspomniany wyżej \emph{overlay} systemu plików.

Kompletna konfiguracja \emph{Buildroot'a} dla systemu użytkowego znajduje się w załączonym pliku \texttt{.config\_users}.

%------------------------------------------------------------------------------

\subsection{Konfiguracja systemu administracyjnego}

Aby skonfigurować system administracyjny zgodnie z wymaganiami przedstawionymi w rozdziale~\ref{task}, należy w konfiguracji \emph{Buildroot'a} (tj. po wydaniu komendy np.~\texttt{make menuconfig}) wybrać m.in.:
\begin{itemize}
	\item Narzędzia do obsługi systemów plików:
	\begin{itemize}
		\item \texttt{BR2\_PACKAGE\_E2FSPROGS},
		\item \texttt{BR2\_PACKAGE\_E2FSPROGS\_RESIZE2FS},
		\item \texttt{BR2\_PACKAGE\_DOSFSTOOLS\_MKFS\_FAT},
		\item \texttt{BR2\_PACKAGE\_DOSFSTOOLS\_FSCK\_FAT},
		\item \texttt{BR2\_PACKAGE\_DOSFSTOOLS\_MKFS\_FAT},
	\end{itemize}
	\item Narzędzia do (de)kompresji:
	\begin{itemize}
		\item \texttt{BR2\_PACKAGE\_ZIP},
		\item \texttt{BR2\_PACKAGE\_UNRAR},
	\end{itemize}
	\item Wygodny edytor tekstu -- \texttt{BR2\_PACKAGE\_VIM},
	\item \texttt{BR2\_PACKAGE\_BUSYBOX\_SHOW\_OTHERS},
	\item Narzędzie \texttt{mkimage} generujące obrazy dla \emph{U-boot'a} -- \texttt{BR2\_PACKAGE\_UBOOT\_TOOLS\_MKIMAGE},
	\item Serwer SSH -- \texttt{BR2\_PACKAGE\_DROPBEAR},
	\item Generowanie \texttt{rootfs.tar} -- \texttt{BR2\_TARGET\_ROOTFS\_TAR}.
\end{itemize}
Niektóre z powyższych pakietów okazały się nadmiarowe (np. obsługa \texttt{zip}'a i \texttt{rar}'a), jednak zostały zawarte w załączonym pliku \texttt{.config}, ponieważ przed zrealizowaniem niniejszego laboratorium, autorzy nie mieli pewności czy narzędzia te nie okażą się przydatne.

Kompletna konfiguracja \emph{Buildroot'a} dla systemu administracyjnego znajduje się w załączonym pliku \texttt{.config}.

%------------------------------------------------------------------------------

\section{Załącznik}

Do niniejszego raportu załączono archiwum:
\begin{center}
\texttt{lab0\labnumber\_BezaSpychala.tar.gz},
\end{center}
o sumie kontrolnej MD5:
\begin{center}
\texttt{d77a3360871370fd1370f42cd09e60f1}
\end{center}
które jest minimalnym zestawem plików, pozwalającym odtworzyć i zweryfikować projekt zgodnie z niżej opisaną procedurą:
\begin{enumerate}
\item W dowolnym katalogu, np. w \path{/tmp/test}, rozpakować \emph{Buildroot'a}:
\begin{center}
\texttt{tar xjvf buildroot-2016.02.tar.bz2}
\end{center}

Dla ustalenia uwagi, niech rozpakowany \emph{Buildroot} będzie w~katalogu:
\begin{center}
\path{/tmp/test/buildroot_2016.2}
\end{center}

\item W katalogu \path{/tmp/test} należy rozpakować archiwum \texttt{lab0\labnumber\_BezaSpychala.tar.gz}:
\begin{center}
\texttt{tar xzvf lab0\labnumber\_BezaSpychala.tar.gz}
\end{center}
\end{enumerate}

Po rozpakowaniu powstanie następująca struktura plików i katalogów:
\dirtree{%
.1 /.
.2 boot.txt.
.2 config.
.3 overlay.
.4 etc.
.5 ifplugd.
.6 ifplugd.action.
.6 ifplugd.conf.
.5 init.d.
.6 S50filesrv.
.6 ifplugd.
.4 root.
.5 server.py.
.5 static.
.6 index.html.
.3 users.txt.
.2 .config.
.2 .config\_users.
}
Pliki te należy przenieść do odpowiadających katalogów w drzewie plików~\emph{Buildroot'a}.

%------------------------------------------------------------------------------

\bibliographystyle{plain}
\bibliography{bibliography}

\end{document}
