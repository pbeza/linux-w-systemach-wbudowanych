\documentclass{article}

\usepackage[OT4,plmath]{polski}
\usepackage[T1]{fontenc}
\usepackage[utf8]{inputenc}
\usepackage{hyperref}
\usepackage{indentfirst}
\usepackage{graphicx}
\usepackage{textcomp}
\usepackage{parskip}
%\usepackage[fixlanguage]{babelbib}
%\selectbiblanguage{polish}
%\setlength{\parindent}{24pt}
%\linespread{1.1}

\frenchspacing

\newenvironment{textfile}{\ttfamily}{\par}

%------------------------------------------------------------------------------

\title{Linux w systemach wbudowanych\\Laboratorium \#1}

\author{\href{mailto:bezap@student.mini.pw.edu.pl}{Patryk \textsc{Bęza}}, \href{mailto:spychalam@student.mini.pw.edu.pl}{Maciej \textsc{Spychała}}}

\date{\today}

\begin{document}

\maketitle

\begin{center}
\begin{tabular}{ll}
Data laboratorium 1A: & 17 III 2016\\
Data laboratorium 1B: & 24 III 2016\\
Wykładowca: & \href{mailto:wzab@ise.pw.edu.pl}{dr inż. Wojciech Zabołotny}
\end{tabular}
\end{center}

%\begin{abstract}
%Abstract text
%\end{abstract}

%------------------------------------------------------------------------------

\section{Zadanie}
\label{task}

Zadaniem laboratorium jest skonfigurowanie \emph{\href{https://buildroot.org/}{Buildroot}'a} i \emph{\href{https://www.raspberrypi.org/}{Raspberry~Pi}} tak, aby \emph{Raspberry~Pi}:

\begin{enumerate}
\item Automatycznie łączył się z siecią, używając protokołu DHCP do otrzymania parametrów sieci. Jeśli kabel sieciowy nie jest podłączony, połączenie powinno zostać zestawione automatycznie po podłączeniu kabla. Odłączenie kabla powinno powodować wyłączenie połączenia.
\item Miał przydzieloną nazwę systemu w postaci \texttt{nazwisko1\_nazwisko2}, czyli w tym przypadku: \texttt{Beza\_Spychala}.
\item Automatycznie aktualizował czas systemowy, wykorzystując do tego klienta NTP (najlepiej z wykorzystaniem \emph{NTP server pool}).
\item Miał zainstalowany dowolny serwer HTTP udostępniający statyczne strony WWW, np.~serwer~\emph{Apache}.
\item Miał poza użytkownikiem \texttt{root}, użytkownika domyślnego, którego hasło jest ustawiane przy starcie systemu.
\end{enumerate}

%------------------------------------------------------------------------------

\section{Konfiguracja}

Poniżej opisano kroki jakie trzeba podjąć, aby skonfigurować \emph{Buildroot'a} i \emph{Raspberry~Pi} tak, aby spełnić wymagania zadania opisane w rozdziale~\ref{task}.

Opcje, o których mowa poniżej, należy włączyć w konfiguracji \emph{Buildroot'a} po uruchomieniu komendy~\texttt{make menuconfig}. Poniżej założono, że korzystamy z \emph{Buildroot'a} w wersji \emph{2016.02}.

%------------------------------------------------------------------------------

\subsection{Konfiguracja sieci}

Aby włączyć obsługę DHCP, należy wybrać opcję:
\begin{center}
\texttt{Target packages \textrightarrow\ Networking applications \textrightarrow\ dhcpcd}
\end{center}

\noindent Do skonfigurowania sieci w taki sposób, aby \emph{Raspberry~Pi} łączyło się z siecią automatycznie po podłączeniu kabla sieciowego, można włączyć pakiet~\texttt{eudev} wybierając:
\begin{center}
\texttt{System configuration \textrightarrow\ /dev management\ \textrightarrow\\Dynamic using devtmpfs + eudev}
\end{center}
a następnie:
\begin{center}
\texttt{Target packages \textrightarrow\ Hardware handling \textrightarrow\ eudev \textrightarrow\\Enable rules generator}
\end{center}
oraz:
\begin{center}
\texttt{Target packages \textrightarrow\ Hardware handling \textrightarrow\ eudev \textrightarrow\\Enable hwdb installation}
\end{center}

\noindent Dzięki włączeniu \texttt{eudev}, możemy skonfigurować automatyczne łączenie z siecią dodając regułę do \href{https://github.com/maximeh/buildroot/blob/master/support/scripts/mkusers}{\texttt{/etc/udev/rules.d}}.

Po dodaniu reguł należy uruchomić \texttt{udevadm control -{}-reload-rules}, aby system załadował dodane przez nas reguły.

%------------------------------------------------------------------------------

\subsection{Nazwa systemu}

Aby zmienić nazwę systemu należy wybrać:

\begin{center}
\texttt{System configuration \textrightarrow\ System hostname}
\end{center}

\noindent i ustawić własną nazwę systemu.

%------------------------------------------------------------------------------

\subsection{NTP}

Aby włączyć automatyczną aktualizację czasu systemu przy pomocy protokołu NTP, należy włączyć \emph{daemona} NTP wybierając:

\begin{center}
\texttt{Target packages \textrightarrow\ Networking applications \textrightarrow\ ntp \textrightarrow\ ntpd}
\end{center}

%------------------------------------------------------------------------------

\subsection{Serwer HTTP}

Aby zainstalować serwer HTTP, serwujący proste, statyczne strony WWW, włączamy opcję aktywującą~\emph{Apache'a}:

\begin{center}
\texttt{Target packages \textrightarrow\ Networking applications \textrightarrow\ apache}
\end{center}

%------------------------------------------------------------------------------

\subsection{Użytkownicy systemu}

Aby dodać użytkownika do systemu, wchodzimy do:

\begin{center}
\texttt{System configuration \textrightarrow\ Path to the users tables}
\end{center}

\noindent i podajemy ścieżkę do pliku ze zdefiniowaną tablica użytkowników systemowych w formacie \href{https://buildroot.org/downloads/manual/manual.html#makeuser-syntax}{opisanym} w dokumentacji \emph{Buildroot'a}. Ścieżka ta zostanie przekazana jako parametr do~\href{https://github.com/maximeh/buildroot/blob/master/support/scripts/mkusers}{\texttt{mkusers}}.

\noindent Przykładowa treść pliku definiującego użytkownika:

\begin{center}
\begin{textfile}
uzyszkodnik -1 users -1 - /home/uzyszkodnik /bin/sh -
\end{textfile}
\end{center}

%------------------------------------------------------------------------------

%\bibliographystyle{plain}
%\bibliography{bibliography}

\end{document}
