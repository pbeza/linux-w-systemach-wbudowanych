\documentclass{article}

\usepackage[OT4,plmath]{polski}
\usepackage[T1]{fontenc}
\usepackage[utf8]{inputenc}
\usepackage{hyperref}
\usepackage{indentfirst}
\usepackage{graphicx}
\usepackage{textcomp}
\usepackage{parskip}
\usepackage{url}
\usepackage{fancyvrb}
\usepackage{dirtree}
\usepackage{listings}
\usepackage[margin=1.5in]{geometry}
\usepackage{float}
\usepackage{microtype}
\usepackage{xcolor}
%\usepackage[fixlanguage]{babelbib}
%\selectbiblanguage{polish}
%\setlength{\parindent}{24pt}
%\linespread{1.1}

\lstset{basicstyle=\ttfamily,
	showstringspaces=false,
	commentstyle=\color{red},
	keywordstyle=\color{blue}
}

\frenchspacing

\newcommand{\labnumber}{4}
\newcommand{\buildrootver}{2016.05}
\newenvironment{ttblock}{\ttfamily}{\par}
\renewcommand{\figurename}{Zdjęcie}

%------------------------------------------------------------------------------

\title{Linux w systemach wbudowanych\\Laboratorium \#\labnumber}

\author{\href{mailto:bezap@student.mini.pw.edu.pl}{Patryk \textsc{Bęza}}, \href{mailto:spychalam@student.mini.pw.edu.pl}{Maciej \textsc{Spychała}}}

\date{\today}

\begin{document}

\maketitle

\begin{center}
\begin{tabular}{ll}
Grupa laboratoryjna: & A\\
Data laboratorium \labnumber A: & 12 V 2016\\
Data laboratorium \labnumber B: & 19 V 2016\\
Wykładowca: & \href{mailto:wzab@ise.pw.edu.pl}{dr inż. Wojciech Zabołotny}
\end{tabular}
\end{center}

%\begin{abstract}
%Abstract text
%\end{abstract}

%------------------------------------------------------------------------------

\section{Zadanie}
\label{task}

Zadaniem laboratorium było zrealizowanie przy pomocy płytki \texttt{Raspberry~Pi}, urządzenia, które będzie wyposażone w złożony interfejs użytkownika:
\begin{itemize}
	\item Przyciski i diody LED powinny być użyte do podstawowej obsługi urządzenia.
	\item Interfejs WWW lub inny interfejs sieciowy powinien być użyty do bardziej zaawansowanych funkcji.
\end{itemize}
Najlepiej, żeby urządzenie przetwarzało dźwięk lub obraz. Stworzenie dokładnej specyfikacji było zadaniem pozostawionym zespołowi. Wybranym zadaniem laboratorium jest następujący temat, oznaczony literą~\textbf{A}:
\begin{description}
\item[Temat A]\hfill\\Wykorzystując kamerę USB, zrealizować system monitorujący obszar widziany przez kamerę. System powinien udostępniać funkcję wysyłania zdjęć robionych co określony czas na zewnętrzny serwer (aby zabezpieczyć nagrany materiał przed zniszczeniem wraz z urządzeniem przez ewentualnego intruza). Operator powinien mieć możliwość zdalnego włączenia funkcji podglądu monitorowanego obszaru ,,na żywo''.
\end{description}

%------------------------------------------------------------------------------

\section{Konfiguracja}

Poniżej opisano kroki jakie trzeba podjąć, aby skonfigurować \texttt{Buildroota} i \texttt{Raspberry~Pi} tak, aby spełnić wymagania zadania opisane w rozdziale~\ref{task}.

%------------------------------------------------------------------------------

\subsection{Konfiguracja wstępna}

Przed rozpoczęciem konfigurowania \texttt{Buildroota}, w taki sposób, aby spełniał warunki zadania, należy skonfigurować podstawowe parametry \texttt{Buildroota} zgodnie z \href{http://wzab.cba.pl/LINSW/poradnik\_laboratorium.pdf}{poradnikiem konfiguracji}, przygotowanym przez Wykładowcę~\cite{www:wzab}, tzn. należy:

\begin{enumerate}
\item Pobrać środowisko \texttt{Buildroot} w najnowszej dostępnej wersji. W dniu pisania niniejszego raportu najnowsza wersja \texttt{Buildroot} to~\texttt{\buildrootver}:

\begin{ttblock}
\$ wget https://buildroot.org/downloads/buildroot-\buildrootver.tar.bz2
\end{ttblock}

\item Rozpakować pobrane środowisko:

\begin{ttblock}
\$ tar xjf buildroot-\buildrootver.tar.bz2
\end{ttblock}

\item Wstępnie skonfigurować do działania z płytką \texttt{Raspberry~Pi}:

\begin{ttblock}
\$ cd buildroot-2016.5\\
\$ make raspberrypi\_defconfig
\end{ttblock}

\item Uruchomić \texttt{make menuconfig}.

\item Ustawić TTY opcją \texttt{BR2\_TARGET\_GENERIC\_GETTY\_PORT}:

\begin{ttblock}
System configuration \textrightarrow\ Run a getty (login prompt) after boot \textrightarrow\ TTY port \textrightarrow\ ttyAMA0
\end{ttblock}

\item Dla poprawy szybkości pobierania zależności oraz dla uniknięcia zablokowania IP laboratoryjnego \emph{gateway'a} przez serwery, z których grupowo pobieramy pakiety, należy ustawić IP lokalnego serwera serwującego część potrzebnego oprogramowania ustawiając opcję \texttt{BR2\_PRIMARY\_SITE}:

\begin{ttblock}
Build options \textrightarrow\ Mirrors and Download locations \textrightarrow\ Primary download site \textrightarrow\ http://192.168.137.24/dl
\end{ttblock}

\sloppy
\item Wyłączyć obsługę modelu \emph{'hard' floating point} opcjami \texttt{BR2\_ARM\_EABI} i \texttt{BR2\_GCC\_TARGET\_FLOAT\_ABI="softfp"}:

\begin{ttblock}
Target options \textrightarrow\ Target ABI \textrightarrow\ EABI
\end{ttblock}
\textbf{Uwaga:} Wyłączenie obsługi modelu \emph{'hard' floating point} może niekorzystnie wpływać na szybkość transmisji video, jednak ze względów na konieczność wybór \emph{toolchainu} \texttt{Sourcery CodeBench ARM 2014.05}, wyłączamy obsługę tego modelu.

\item Zmienić \emph{Toolchain} opcjami \texttt{BR2\_TOOLCHAIN\_EXTERNAL} i\\ \texttt{BR2\_TOOLCHAIN\_EXTERNAL\_CODESOURCERY\_ARM}:

\begin{ttblock}
Toolchain \textrightarrow\ Toolchain type \textrightarrow\ External toolchain\\
Toolchain \textrightarrow\ Toolchain \textrightarrow\ Sourcery CodeBench ARM 2014.05
\end{ttblock}

%\item Dodać \texttt{initramfs} jako rodzaj generowanego obrazu \emph{filesystem'u}, pozostawiając zaznaczony również \texttt{ext2/3/4 variant (ext4)} -- opcja \\\texttt{BR2\_TARGET\_ROOTFS\_INITRAMFS}:
%
%\begin{ttblock}
%Filesystem images \textrightarrow\ initial RAM filesystem linked into linux kernel
%\end{ttblock}
\end{enumerate}

Kolejne kroki konieczne do zrealizowania zadania niniejszego laboratorium opisano w kolejnych podrozdziałach.

%------------------------------------------------------------------------------

\subsection{Pakiety dodatkowe}

Lista wszystkich pakietów \texttt{Buildroot'a} wymaganych do poprawnego skonfigurowania omawianego projektu jest zawarta w pliku \texttt{.config}, dostarczonym jako załącznik do niniejszej dokumentacji. W~szczególności najistotniejsze pakiety, które należy wybrać w \texttt{Buildroot'cie} (lub które zostały wybrane jako zależność do wybranych pakietów), to:
\begin{verbatim}
ffmpeg, ffserver, Python, libjpeg, libv4l, dropbear, ifplugd, lftp,
ntpd, ntpdate, rsync, tornado, libdaemon, libevent, libffi, ncurses
\end{verbatim}
Warto zauważyć, że do \emph{streamingu} video został użyty \texttt{ffserver}.

%------------------------------------------------------------------------------

\subsection{Sterownik kamery}

W projekcie użyto kamery internetowej \texttt{Creative~Live!~Cam~Connect~HD~1080~(VF0760)}, przedstawionej na rysunku~\ref{fig:creative}, która co prawda ma odpowiedni moduł w domyślnym kernelu, ale nie jest on automatycznie ładowany w czasie startu \texttt{Raspberry~Pi}. Ładowanie modułu można wymusić ręcznie za pomocą komendy:
\begin{verbatim}
# modprobe uvcvideo
\end{verbatim}
gdzie \texttt{uvcvideo} to nazwa modułu obsługującego wykorzystaną kamerę. W wersji końcowej projektu zautomatyzowano ten proces, tak aby przy każdym uruchomieniu \texttt{Raspberry~Pi} moduł kamery był ładowany. W tym celu należało utworzyć plik \path{/etc/modules}, a w nim umieścić nazwę modułu, czyli w naszym przypadku, dla kamery \texttt{Creative~VF0760} jest to moduł~\texttt{uvcvideo}.
\begin{figure}[h]
	\centering
	\includegraphics[width=0.6\textwidth]{img/creative}
	\caption{Kamera~\texttt{Creative~VF0760} wykorzystana w projekcie}
	\label{fig:creative}
\end{figure}

%------------------------------------------------------------------------------

\subsection{Konfiguracja \texttt{ffserver}}

Konfiguracja serwera \texttt{ffserver} została zapisana w pliku konfiguracynym \path{/etc/ffserver.conf}. W konfiguracji umieszczono m.in. wpisy dotyczące:
\begin{itemize}
	\item portu, na którym serwer działa -- wybrano port \texttt{Port~8080},
	\item maksymalną liczbę równoczesnych połączeń \texttt{HTTP} -- wybrano \texttt{MaxHTTPConnections~30},
	\item maksymalną liczbę równoczesnych żądań\footnote{\href{https://www.ffmpeg.org/ffserver.html}{Dokumentacja \texttt{ffserver'a}} mówi, że \texttt{MaxHTTPConnections} musi być zdefiniowane przed \texttt{MaxClients} oraz musi zachodzić nierówność $\texttt{MaxClients}\leq\texttt{MaxHTTPConnections}$.} połączeń -- wybrano \texttt{MaxClients~20},
	\item sposobu kodowania \emph{streamingu} -- wybrano \texttt{mjpeg},
	\item wymiarów nagrania wyrażonego w pikselach -- wybrano \texttt{VideoSize~320x240},
	\item \emph{bit rate'u} -- wybrano \texttt{VideoBitRate~2000} -- wyrażone w kbit/s,
	\item \emph{video frame rate} -- wybrano \texttt{VideoFrameRate 25} -- wyrażone w ilości ramek na sekundę,
	\item maksymalnej wielkości \emph{feed file} dla \emph{streamingu} -- wybrano \texttt{FileMaxSize~5M} -- wyrażone w megabajtach.
\end{itemize}

Do uruchomienia \emph{streamingu} stworzono skrypt, którego najważniejszą częścią są komendy (TODO):
\begin{verbatim}
$ ffserver &
$ ffmpeg -framerate 25 -video_size -i /dev/video0 -f v4l2 /tmp/cam1.ffm
\end{verbatim}

%------------------------------------------------------------------------------

\subsection{Interfejs WWW}
\label{sec:www}

W celu umożliwienia kontrolowania działania \texttt{Raspberry~Pi}, zainstalowano serwer WWW \texttt{tornado}, na którym skonfigurowano interfejs użytkownika z autoryzacją przez prosty formularz \texttt{HTML}.
\begin{figure}[H]
\centering
\begin{tabular}{r|l}
Login: & \texttt{admin}\\
\hline
Hasło: & \texttt{admin1}\\
\end{tabular}
\caption{Dane logowania do interfejsu WWW}
\end{figure}
Po zalogowaniu, użytkownik ma możliwość na żądanie przerwać oraz ponownie uruchomić \emph{streaming} video. Plik konfiguracyjny \texttt{tornado} ma nazwę \path{server.py}.

%------------------------------------------------------------------------------

\subsection{Skrypty \texttt{init.d}}

Aby uruchomić \texttt{ffserver} po każdym starcie \texttt{Raspberry~Pi}, stworzono skrypt, który umieszczono w katalogu \path{/etc/init.d}. Ten sam skrypt jest używany przy wymuszonym starcie \emph{streaming'u} przez interfejs WWW, opisany w rozdziale~\ref{sec:www}.

Ponadto, w \path{/etc/init.d} umieszczono skrypt synchronizujący datę za pomocą protokołu~\texttt{NTP}.

%------------------------------------------------------------------------------

\subsection{Wpisy w \texttt{cron}}

Do zapewnienia regularnej synchronizacji czasu przez protokół \texttt{NTP} oraz w celu regularnego \emph{uploadu} regularnie robionych zdjęć na serwer \texttt{FTP}, wykorzystano \emph{demona} systemowego \texttt{cron}. Konfiguracja uruchamiania \texttt{cron'a} została zachowana w pliku \path{/etc/crontab} i sprowadza się do uruchamiania co minutę skryptu \path{/root/sync.sh} o zawartości przedstawionej na listingu~\ref{lst:sync}.
\begin{lstlisting}[caption={Zawartość pliku \texttt{/root/sync.sh}},label=lst:sync,language=bash,frame=single,breaklines,captionpos=b]
#!/bin/sh
SRC=/root/photos
DEST=/tmp
#DEST=root@127.0.0.1:/tmp
rsync -a --remove-source-files \$SRC \$DEST
\end{lstlisting}

%------------------------------------------------------------------------------

\section{Załącznik}

Do niniejszego raportu załączono archiwum:
\begin{center}
\texttt{lab0\labnumber\_BezaSpychala.tar.gz}
\end{center}
o sumie kontrolnej MD5:
\begin{center}
\texttt{11dd21e8c3104588546c71f6fc28c674}
\end{center}
które jest minimalnym zestawem plików, pozwalającym odtworzyć i zweryfikować projekt zgodnie z niżej opisaną procedurą:
\begin{enumerate}
\item Rozpakować \texttt{Buildroota} do dowolnego katalogu:
\begin{center}
\texttt{tar xjvf buildroot-\buildrootver.tar.bz2}
\end{center}

\item W dowolnym katalogu rozpakować archiwum \texttt{lab0\labnumber\_BezaSpychala.tar.gz}:
\begin{center}
\texttt{tar xzvf lab0\labnumber\_BezaSpychala.tar.gz}
\end{center}
\end{enumerate}
\newpage
Po rozpakowaniu \texttt{lab0\labnumber\_BezaSpychala.tar.gz} powstanie następująca struktura plików i katalogów:
\dirtree{%
.1 lab0\labnumber\_BezaSpychala.
.2 config.
.3 overlay.
.4 etc.
.5 crontab.
.5 ffserver.conf.
.5 ifplugd.
.6 ifplugd.action.
.6 ifplugd.conf.
.5 init.d.
.6 crond.
.6 ffserver.
.6 ifplugd.
.6 photos.
.6 S18modules.
.6 S80crond.
.6 S89udhcp.
.6 S90ifplugd.
.6 S98serverwww.
.6 S90photos.
.6 serverwww.
.5 modules.
.4 root.
.5 photos.
.5 server.py.
.5 static.
.6 index.html.
.6 jquery-2.1.3.min.js.
.6 jquery.clipchamp.mjpeg.player.js.
.6 login.html.
.5 sync.sh.
.3 users.txt.
.2 .config.
}
Pliki te należy przenieść do odpowiadających katalogów w drzewie plików~\texttt{Buildroota}, zgodnie z niniejszą dokumentacją.

%------------------------------------------------------------------------------

\newpage
\bibliographystyle{ieeetr}
\bibliography{bibliography}

\end{document}
